%-------------------------
% Resume - Professional Technical Format
% Author: Wang Jintian (Andrew)
% License: MIT
%-------------------------

\documentclass[letterpaper,11pt]{article}

\usepackage{latexsym}
\usepackage[empty]{fullpage}
\usepackage{titlesec}
\usepackage{marvosym}
\usepackage[usenames,dvipsnames]{color}
\usepackage{verbatim}
\usepackage{enumitem}
\usepackage[hidelinks]{hyperref}
\usepackage{fancyhdr}
\usepackage[english]{babel}
\usepackage{tabularx}
\usepackage{fontawesome5}
\usepackage{multicol}
\setlength{\multicolsep}{-3.0pt}
\setlength{\columnsep}{-1pt}
\input{glyphtounicode}

%----------FONT OPTIONS----------
\pagestyle{fancy}
\fancyhf{}
\fancyfoot{}
\renewcommand{\headrulewidth}{0pt}
\renewcommand{\footrulewidth}{0pt}

% Adjust margins
\addtolength{\oddsidemargin}{-0.6in}
\addtolength{\evensidemargin}{-0.5in}
\addtolength{\textwidth}{1.19in}
\addtolength{\topmargin}{-.7in}
\addtolength{\textheight}{1.4in}

\urlstyle{same}

\raggedbottom
\raggedright
\setlength{\tabcolsep}{0in}

% Sections formatting
\titleformat{\section}{
  \vspace{-4pt}\scshape\raggedright\large\bfseries
}{}{0em}{}[\color{black}\titlerule \vspace{-5pt}]

% Ensure that generate pdf is machine readable/ATS parsable
\pdfgentounicode=1

%-------------------------
% Custom commands
\newcommand{\resumeItem}[1]{
  \item\small{
    {#1 \vspace{-2pt}}
  }
}

\newcommand{\classesList}[4]{
    \item\small{
        {#1 #2 #3 #4 \vspace{-2pt}}
  }
}

\newcommand{\resumeSubheading}[4]{
  \vspace{-2pt}\item
    \begin{tabular*}{1.0\textwidth}[t]{l@{\extracolsep{\fill}}r}
      \textbf{#1} & \textbf{\small #2} \\
      \textit{\small#3} & \textit{\small #4} \\
    \end{tabular*}\vspace{-7pt}
}

\newcommand{\resumeSubSubheading}[2]{
    \item
    \begin{tabular*}{1.0\textwidth}{l@{\extracolsep{\fill}}r}
      \textit{\small#1} & \textit{\small #2} \\
    \end{tabular*}\vspace{-7pt}
}

\newcommand{\resumeProjectHeading}[2]{
    \item
    \begin{tabular*}{1.001\textwidth}{l@{\extracolsep{\fill}}r}
      \small#1 & \textbf{\small #2}\\
    \end{tabular*}\vspace{-7pt}
}

\newcommand{\resumeSubItem}[1]{\resumeItem{#1}\vspace{-4pt}}

\renewcommand\labelitemi{$\vcenter{\hbox{\tiny$\bullet$}}$}
\renewcommand\labelitemii{$\vcenter{\hbox{\tiny$\bullet$}}$}

\newcommand{\resumeSubHeadingListStart}{\begin{itemize}[leftmargin=0.0in, label={}]}
\newcommand{\resumeSubHeadingListEnd}{\end{itemize}}
\newcommand{\resumeItemListStart}{\begin{itemize}}
\newcommand{\resumeItemListEnd}{\end{itemize}\vspace{-5pt}}

%-------------------------------------------
%%%%%%  RESUME STARTS HERE  %%%%%%%%%%%%%%%%%%%%%%%%%%%%

\begin{document}

%----------HEADING----------
\begin{center}
    {\Huge \scshape Wang Jintian (Andrew)} \\ \vspace{1pt}
    \small
    \raisebox{-0.1\height}\faPhone\ +1-xxx-xxx-xxxx ~
    \href{mailto:your.email@example.com}{\raisebox{-0.2\height}\faEnvelope\  \underline{your.email@example.com}} ~
    \href{https://linkedin.com/in/yourprofile}{\raisebox{-0.2\height}\faLinkedin\ \underline{linkedin.com/in/yourprofile}} ~
    \href{https://github.com/yourusername}{\raisebox{-0.2\height}\faGithub\ \underline{github.com/yourusername}}
    \vspace{-8pt}
\end{center}

%-----------EDUCATION-----------
\section{Education}
  \resumeSubHeadingListStart
    \resumeSubheading
      {Your University Name}{Expected Graduation: Month 20XX}
      {Bachelor of Science in Computer Science, GPA: 3.XX/4.0}{City, State/Country}
      \resumeItemListStart
        \resumeItem{\textbf{Relevant Coursework:} Data Structures, Algorithms, Machine Learning, Database Systems, Web Development, Software Engineering}
      \resumeItemListEnd
  \resumeSubHeadingListEnd

%-----------TECHNICAL SKILLS-----------
\section{Technical Skills}
 \begin{itemize}[leftmargin=0.15in, label={}]
    \small{\item{
     \textbf{Languages}{: Python, TypeScript, JavaScript, SQL, HTML/CSS, LaTeX} \\
     \textbf{AI/ML Frameworks}{: LangGraph, LangChain, Ollama, RAG Systems, Vector Databases, Embeddings} \\
     \textbf{Backend Frameworks}{: FastAPI, RESTful APIs, WebSocket, Event-Driven Architecture} \\
     \textbf{Frontend Frameworks}{: React 19, Vite, Tailwind CSS, React Hooks, State Management} \\
     \textbf{Databases \& Storage}{: PostgreSQL, Redis, ChromaDB, Vector Embeddings, Document Stores} \\
     \textbf{DevOps \& Tools}{: Docker, Docker Compose, Git, GitHub, LangSmith, Ollama} \\
     \textbf{Architecture Patterns}{: State Machines, Router Pattern, Event Streaming, Microservices, RAG Architecture} \\
    }}
 \end{itemize}
 \vspace{-16pt}

%-----------PROJECTS-----------
\section{Projects}
    \resumeSubHeadingListStart
      \resumeProjectHeading
          {\textbf{Intelligent Knowledge Base Q\&A System} $|$ \emph{Python, LangGraph, FastAPI, React, TypeScript, Ollama}}{Dec 2024}
          \resumeItemListStart
            \resumeItem{Architected a \textbf{LangGraph state machine workflow} orchestrating 3 distinct conversation modes (RAG, GPT, DeepResearch) with conditional routing and event-driven streaming, enabling flexible AI-powered question answering across multiple knowledge sources}
            \resumeItem{Engineered a \textbf{multi-round DeepResearch system} with AI-driven reflection mechanism that iteratively refines search queries across 3 loops, achieving 85\%+ confidence scores through automated quality assessment and gap analysis}
            \resumeItem{Implemented \textbf{custom knowledge base upload-to-query pipeline} supporting 10+ document formats (PDF, DOCX, TXT), processing documents into vector embeddings with semantic chunking and enabling similarity-based retrieval}
            \resumeItem{Designed a \textbf{knowledge base router abstraction layer} enabling seamless switching between CNB API, Wikipedia, and custom knowledge bases, reducing code duplication by 60\% and improving system extensibility}
            \resumeItem{Built \textbf{real-time event streaming architecture} using FastAPI WebSocket and custom LangGraph events, providing live progress updates during multi-round research with <100ms latency for enhanced user experience}
            \resumeItem{Optimized \textbf{multi-model collaboration strategy} leveraging lightweight models (Qwen2.5:3b) for query generation and capable models (Qwen2.5:7b) for analysis, reducing average query latency by 60\% while maintaining quality}
            \resumeItem{Developed \textbf{full-stack web interface} with React 19 and TypeScript, featuring conversation history management, source citation tracking with clickable references, and dynamic knowledge base selection with animated UI components}
          \resumeItemListEnd
          \vspace{-13pt}

      % Add more projects here as needed
      % \resumeProjectHeading
      %     {\textbf{Another Project Name} $|$ \emph{Tech Stack}}{Month Year}
      %     \resumeItemListStart
      %       \resumeItem{Project description and achievements...}
      %     \resumeItemListEnd
    \resumeSubHeadingListEnd
\vspace{-15pt}

%-----------EXPERIENCE-----------
\section{Experience}
  \resumeSubHeadingListStart

    % Uncomment and fill in your experience
    % \resumeSubheading
    %   {Company Name}{Month Year -- Month Year}
    %   {Position Title}{City, State/Country}
    %   \resumeItemListStart
    %     \resumeItem{Description of your responsibilities and achievements...}
    %     \resumeItem{Another achievement with quantifiable results...}
    %   \resumeItemListEnd

    % For students with no experience, you can add:
    \resumeSubheading
      {Open Source Contributions \& Personal Projects}{2024 -- Present}
      {Independent Developer}{Remote}
      \resumeItemListStart
        \resumeItem{Developed multiple full-stack applications using modern frameworks including LangGraph, FastAPI, and React}
        \resumeItem{Implemented AI-powered systems integrating large language models with RAG architectures}
      \resumeItemListEnd

  \resumeSubHeadingListEnd
\vspace{-16pt}

%-----------ADDITIONAL SECTIONS-----------
% \section{Certifications}
%  \begin{itemize}[leftmargin=0.15in, label={}]
%     \small{\item{
%      \textbf{Certification Name}{: Issuing Organization, Month Year} \\
%      \textbf{Another Certification}{: Issuing Organization, Month Year} \\
%     }}
%  \end{itemize}
%  \vspace{-16pt}

% \section{Awards \& Honors}
%  \begin{itemize}[leftmargin=0.15in, label={}]
%     \small{\item{
%      \textbf{Award Name}{: Description, Year} \\
%      \textbf{Another Award}{: Description, Year} \\
%     }}
%  \end{itemize}
%  \vspace{-16pt}

\end{document}
