%-------------------------
% 简历 - 专业技术格式(中文版)
% 作者: 王金田 (Andrew Wang)
% 协议: MIT
%-------------------------

\documentclass[letterpaper,11pt]{article}

\usepackage{latexsym}
\usepackage[empty]{fullpage}
\usepackage{titlesec}
\usepackage{marvosym}
\usepackage[usenames,dvipsnames]{color}
\usepackage{verbatim}
\usepackage{enumitem}
\usepackage[hidelinks]{hyperref}
\usepackage{fancyhdr}
\usepackage[english]{babel}
\usepackage{tabularx}
\usepackage{fontawesome5}
\usepackage{multicol}
\usepackage{xeCJK}
\setlength{\multicolsep}{-3.0pt}
\setlength{\columnsep}{-1pt}
% glyphtounicode not needed for XeLaTeX

% 中文字体设置(优先级顺序:系统默认字体)
% macOS用户会使用PingFang/STSong,Windows用户会使用SimSun/SimHei
\setCJKmainfont{STSong}[
    BoldFont=STHeiti,
    ItalicFont=STKaiti
]
\setCJKsansfont{STHeiti}
\setCJKmonofont{STFangsong}

%----------字体选项----------
\pagestyle{fancy}
\fancyhf{}
\fancyfoot{}
\renewcommand{\headrulewidth}{0pt}
\renewcommand{\footrulewidth}{0pt}

% 调整页边距
\addtolength{\oddsidemargin}{-0.6in}
\addtolength{\evensidemargin}{-0.5in}
\addtolength{\textwidth}{1.19in}
\addtolength{\topmargin}{-.7in}
\addtolength{\textheight}{1.4in}

\urlstyle{same}

\raggedbottom
\raggedright
\setlength{\tabcolsep}{0in}

% 章节格式
\titleformat{\section}{
  \vspace{-4pt}\scshape\raggedright\large\bfseries
}{}{0em}{}[\color{black}\titlerule \vspace{-5pt}]

% XeLaTeX automatically generates Unicode-compatible PDFs

%-------------------------
% 自定义命令
\newcommand{\resumeItem}[1]{
  \item\small{
    {#1 \vspace{-2pt}}
  }
}

\newcommand{\classesList}[4]{
    \item\small{
        {#1 #2 #3 #4 \vspace{-2pt}}
  }
}

\newcommand{\resumeSubheading}[4]{
  \vspace{-2pt}\item
    \begin{tabular*}{1.0\textwidth}[t]{l@{\extracolsep{\fill}}r}
      \textbf{#1} & \textbf{\small #2} \\
      \textit{\small#3} & \textit{\small #4} \\
    \end{tabular*}\vspace{-7pt}
}

\newcommand{\resumeSubSubheading}[2]{
    \item
    \begin{tabular*}{1.0\textwidth}{l@{\extracolsep{\fill}}r}
      \textit{\small#1} & \textit{\small #2} \\
    \end{tabular*}\vspace{-7pt}
}

\newcommand{\resumeProjectHeading}[2]{
    \item
    \begin{tabular*}{1.001\textwidth}{l@{\extracolsep{\fill}}r}
      \small#1 & \textbf{\small #2}\\
    \end{tabular*}\vspace{-7pt}
}

\newcommand{\resumeSubItem}[1]{\resumeItem{#1}\vspace{-4pt}}

\renewcommand\labelitemi{$\vcenter{\hbox{\tiny$\bullet$}}$}
\renewcommand\labelitemii{$\vcenter{\hbox{\tiny$\bullet$}}$}

\newcommand{\resumeSubHeadingListStart}{\begin{itemize}[leftmargin=0.0in, label={}]}
\newcommand{\resumeSubHeadingListEnd}{\end{itemize}}
\newcommand{\resumeItemListStart}{\begin{itemize}}
\newcommand{\resumeItemListEnd}{\end{itemize}\vspace{-5pt}}

%-------------------------------------------
%%%%%%  简历开始  %%%%%%%%%%%%%%%%%%%%%%%%%%%%

\begin{document}

%----------标题----------
\begin{center}
    {\Huge \scshape 王金田 (Andrew Wang)} \\ \vspace{1pt}
    \small
    \raisebox{-0.1\height}\faPhone\ +86-xxx-xxxx-xxxx ~
    \href{mailto:your.email@example.com}{\raisebox{-0.2\height}\faEnvelope\  \underline{your.email@example.com}} ~
    \href{https://linkedin.com/in/yourprofile}{\raisebox{-0.2\height}\faLinkedin\ \underline{linkedin.com/in/yourprofile}} ~
    \href{https://github.com/yourusername}{\raisebox{-0.2\height}\faGithub\ \underline{github.com/yourusername}}
    \vspace{-8pt}
\end{center}

%-----------教育背景-----------
\section{教育背景}
  \resumeSubHeadingListStart
    \resumeSubheading
      {您的大学名称}{预计毕业时间:20XX年XX月}
      {计算机科学学士,GPA: 3.XX/4.0}{城市,国家}
      \resumeItemListStart
        \resumeItem{\textbf{相关课程:}数据结构、算法、机器学习、数据库系统、Web开发、软件工程}
      \resumeItemListEnd
  \resumeSubHeadingListEnd

%-----------专业技能-----------
\section{专业技能}
 \begin{itemize}[leftmargin=0.15in, label={}]
    \small{\item{
     \textbf{编程语言}{: Python, TypeScript, JavaScript, SQL, HTML/CSS, LaTeX} \\
     \textbf{AI/ML 框架}{: LangGraph, LangChain, Ollama, RAG系统, 向量数据库, Embeddings} \\
     \textbf{后端框架}{: FastAPI, RESTful APIs, WebSocket, 事件驱动架构} \\
     \textbf{前端框架}{: React 19, Vite, Tailwind CSS, React Hooks, 状态管理} \\
     \textbf{数据库与存储}{: PostgreSQL, Redis, ChromaDB, 向量嵌入, 文档存储} \\
     \textbf{DevOps 与工具}{: Docker, Docker Compose, Git, GitHub, LangSmith, Ollama} \\
     \textbf{架构模式}{: 状态机, 路由模式, 事件流, 微服务, RAG架构} \\
    }}
 \end{itemize}
 \vspace{-16pt}

%-----------项目经历-----------
\section{项目经历}
    \resumeSubHeadingListStart
      \resumeProjectHeading
          {\textbf{智能知识库问答系统} $|$ \emph{Python, LangGraph, FastAPI, React, TypeScript, Ollama}}{2024年12月}
          \resumeItemListStart
            \resumeItem{架构设计了\textbf{LangGraph状态机工作流},编排三种对话模式(RAG模式、GPT模式、深度研究模式),实现条件路由和事件驱动流式传输,支持跨多个知识源的灵活AI问答}
            \resumeItem{工程化实现了\textbf{多轮深度研究系统},集成AI驱动的反思机制,通过3轮迭代自动优化搜索查询,借助自动化质量评估和差距分析实现85\%+的置信度评分}
            \resumeItem{实现了\textbf{自定义知识库上传至查询全流程},支持10+种文档格式(PDF、DOCX、TXT),通过语义分块将文档处理为向量嵌入,实现基于相似度的检索}
            \resumeItem{设计了\textbf{知识库路由抽象层},实现CNB API、Wikipedia和自定义知识库之间的无缝切换,减少60\%代码重复,提升系统扩展性}
            \resumeItem{构建了\textbf{实时事件流架构},使用FastAPI WebSocket和自定义LangGraph事件,在多轮研究过程中提供<100ms延迟的实时进度更新,显著提升用户体验}
            \resumeItem{优化了\textbf{多模型协作策略},利用轻量级模型(Qwen2.5:3b)生成查询,功能强大的模型(Qwen2.5:7b)进行分析,在保证质量的同时将平均查询延迟降低60\%}
            \resumeItem{开发了\textbf{全栈Web界面},使用React 19和TypeScript实现,具备对话历史管理、可点击的来源引用追踪、动态知识库选择及动画UI组件}
          \resumeItemListEnd
          \vspace{-13pt}

      % 如需添加更多项目,在此处添加
      % \resumeProjectHeading
      %     {\textbf{其他项目名称} $|$ \emph{技术栈}}{年月}
      %     \resumeItemListStart
      %       \resumeItem{项目描述和成就...}
      %     \resumeItemListEnd
    \resumeSubHeadingListEnd
\vspace{-15pt}

%-----------工作经历-----------
\section{工作经历}
  \resumeSubHeadingListStart

    % 取消注释并填写您的工作经历
    % \resumeSubheading
    %   {公司名称}{年月 -- 年月}
    %   {职位名称}{城市,国家}
    %   \resumeItemListStart
    %     \resumeItem{您的职责和成就描述...}
    %     \resumeItem{另一项可量化的成就...}
    %   \resumeItemListEnd

    % 对于无工作经验的学生,可以添加:
    \resumeSubheading
      {开源贡献与个人项目}{2024年 -- 至今}
      {独立开发者}{远程}
      \resumeItemListStart
        \resumeItem{使用现代框架开发多个全栈应用,包括LangGraph、FastAPI和React}
        \resumeItem{实现了集成大语言模型和RAG架构的AI驱动系统}
      \resumeItemListEnd

  \resumeSubHeadingListEnd
\vspace{-16pt}

%-----------其他部分-----------
% \section{证书}
%  \begin{itemize}[leftmargin=0.15in, label={}]
%     \small{\item{
%      \textbf{证书名称}{: 颁发机构,年月} \\
%      \textbf{另一个证书}{: 颁发机构,年月} \\
%     }}
%  \end{itemize}
%  \vspace{-16pt}

% \section{奖项与荣誉}
%  \begin{itemize}[leftmargin=0.15in, label={}]
%     \small{\item{
%      \textbf{奖项名称}{: 描述,年份} \\
%      \textbf{另一个奖项}{: 描述,年份} \\
%     }}
%  \end{itemize}
%  \vspace{-16pt}

\end{document}
